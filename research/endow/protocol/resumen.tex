\documentclass[a4paper,10pt]{article}
\usepackage[utf8]{inputenc}
\usepackage{fullpage}
\usepackage{cite}
\usepackage[utf8]{inputenc}
\usepackage{a4wide}
\usepackage{url}
\usepackage{graphicx}
\usepackage{caption}
\usepackage{float} % para que los gr\'aficos se queden en su lugar con [H]
\usepackage{subcaption}
\usepackage{wrapfig}
\usepackage{color}
\usepackage{amsmath} %para escribir funci\'on partida , matrices
\usepackage{amsthm} %para numerar definciones y teoremas
\usepackage[hidelinks]{hyperref} % para inlcuir links dentro del texto
\usepackage{tabu} 
\usepackage{comment}
\usepackage{amsfonts} % \mathbb{N} -> conjunto de los n\'umeros naturales  
\usepackage{enumerate}
\usepackage{listings}
\usepackage[colorinlistoftodos, textsize=small]{todonotes} % Para poner notas en el medio del texto!! No olvidar hacer. 
\usepackage{framed} % Para encuadrar texto. \begin{framed}
\usepackage{csquotes} % Para citar texto \begin{displayquote}
\usepackage{epigraph} % Epigrafe  \epigraph{texto}{\textit{autor}}
\usepackage{authblk}
\usepackage{titlesec}
\usepackage{varioref}
\usepackage{bm} % \bm{\alpha} bold greek symbol
\usepackage{pdfpages} % \includepdf
\usepackage[makeroom]{cancel} % \cancel{} \bcancel{} etc
\usepackage{wrapfig} % \begin{wrapfigure} Pone figura al lado del texto
\usepackage{mdframed}
\usepackage{algorithm}
%\usepackage{quoting}
\usepackage{mathtools}	
\usepackage{tikz}
\usepackage{paracol}

\newcommand{\vm}[1]{\mathbf{#1}}
\newcommand{\N}{\mathcal{N}}
\newcommand{\citel}[1]{\cite{#1}\label{#1}}
\newcommand\hfrac[2]{\genfrac{}{}{0pt}{}{#1}{#2}} %\frac{}{} sin la linea del medio

\newtheorem{midef}{Definition}
\newtheorem{miteo}{Theorem}
\newtheorem{mipropo}{Proposition}

\theoremstyle{definition}
\newtheorem{definition}{Definition}[section]
\newtheorem{theorem}{Theorem}[section]
\newtheorem{proposition}{Proposition}[section]


%http://latexcolor.com/
\definecolor{azul}{rgb}{0.36, 0.54, 0.66}
\definecolor{rojo}{rgb}{0.7, 0.2, 0.116}
\definecolor{rojopiso}{rgb}{0.8, 0.25, 0.17}
\definecolor{verdeingles}{rgb}{0.12, 0.5, 0.17}
\definecolor{ubuntu}{rgb}{0.44, 0.16, 0.39}
\definecolor{debian}{rgb}{0.84, 0.04, 0.33}
\definecolor{dkgreen}{rgb}{0,0.6,0}
\definecolor{gray}{rgb}{0.5,0.5,0.5}
\definecolor{mauve}{rgb}{0.58,0,0.82}

\lstset{
  language=Python,
  aboveskip=3mm,
  belowskip=3mm,
  showstringspaces=true,
  columns=flexible,
  basicstyle={\small\ttfamily},
  numbers=none,
  numberstyle=\tiny\color{gray},
  keywordstyle=\color{blue},
  commentstyle=\color{dkgreen},
  stringstyle=\color{mauve},
  breaklines=true,
  breakatwhitespace=true,
  tabsize=4
}

%opening
\title{}
\author{}

\begin{document}

\maketitle

\section{Introducción}

\paragraph{Sobre el proyecto} ENDOW es una asociación de etnógrafos de todo el mundo que acordaron un protocolo para realizar análisis comparativo-longitudinal de sus propios campos, y el desarrollo de publicaciones colaborativas.

\paragraph{Objetivo} Analizar cómo la estrucutra de la red influencia la desigualdad de las riquezas en sociedades pequeñas.

\paragraph{Datos} Se recolectan datos cada tres años en todas las unidades domesticas sobre: 1) la riqueza en términos amplios; 2) sus redes sociales (de apoyo y afecto); 3) otras variables.

\paragraph{Protocolo} Cuestionario estructurado (10 preguntas) adaptable al campo que deben hacerse cada tres años sobre todas las unidades domésticas (requisito).

\vspace{0.1cm}

Si bien el cuestionario es estructurado, se espera que les etnógrafes adapten la herramienta a las caracterísiticas propias del campo en función de la propia intución y experiencia.

\section{Definiciones}

\paragraph{Comunidad} Es importante definir los límites de la comunidad (o red) de análisis para con el fin de analizar el ``grafo completo''. El criterio puede ser tanto un geográfico, político como las municipalidades u otro. El tamaño mínimo son 30 unidades domésticas.

\paragraph{Unidades domésticas} Las unidades de análisis no son tanto las unidades de residencia como las unidades de intercambio, en el que los recursos sociales y materiales son típicamente compartidos. Es necesario identificar las unidades. Criterios respecto de fusiones y fisiones de unidades domesticas se detallan en el protocolo completo.

\paragraph{Redes}
\begin{itemize}\setlength\itemsep{0cm}
 \item Redes Q: Redes de apoyo material y comportamental
 \item Redes K: Redes de afecto y comunicación
\end{itemize}
 
\paragraph{Informantes} Al menos un informante por unidad de intercambio. Se prioriza la coherencia en la toma de datos. Si se decide tomar consultar a una mujer y un hombre, este criterio debe ser respetado en todas las unidades. No es un problema tener entrevistas adicionales.
\begin{itemize}\setlength\itemsep{0cm}
 \item El cuestionario debe ser tomado por separado a cada informante
 \item
\end{itemize}

\section{Preguntas}

El orden de las preguntas importa.
\begin{enumerate}
 \item Pregunta estandarizada sobre de quién recibió prestamo de dinero (recibir)
 
 \textbf{En caso de necesidad urgente, ¿a quién podrías pedirle prestado el equivalente a una semana de sueldo?}
 
 \item Pregunta estandarizada sobre a quién entregó un prestamo de dinero (entregar)
 
 \textbf{En caso de necesidad urgente, ¿a quién le prestarías el equivalente a una semana de sueldo?}
 
 \item Cuestión de intercambio a nivel de hogar, probablemente compartir alimentos en la mayoría de los sitios o algo comparable en otros lugares (recibir)
 
 Esta pregunta está diseñada para captar los intercambios de recursos a nivel de la unidad de residencia a corto plazo. 
 
 \item Pregunta inversa de intercambio a nivel de hogar (entregar)
 
 Esta pregunta está diseñada para captar los intercambios de recursos a nivel de la unidad de residencia a corto plazo. 
 
 \item Red Q orientada a las mujeres
 
 Esta pregunta requiere que se considere el trabajo productivo de las mujeres.
 Las posibles preguntas que se discutieron en el taller fueron por ejemplo (1) ¿Quién te cuida a los niños (para que puedas trabajar)?, (2) ¿Con quién cooperas en el trabajo agrícola?, y (3) ¿Quién te ayuda a llevar tu negocio mercantil?
 
 \item Red Q orientada a los hombres
 
 Esta pregunta requiere que se considere el trabajo productivo de los hombres.
 
 Algunos ejemplos de preguntas que se debatieron durante el taller son: (1) ¿Con quién cazas?, (2) ¿Con quién mantienes un sistema de riego?, y (3) ¿Quién te ayuda a cuidar de tus animales?
 
 \item Red K orientada a las mujeres
 
 Esta pregunta requiere que consideres los vínculos comunicativos y afectivos de las mujeres. 
 Las posibles preguntas podrían enmarcarse en torno a las visitas, como en ¿A quién ves a veces los domingos por la tarde? 
 
 \item Red K orientada a los hombres
 Esta pregunta requiere que consideres los vínculos comunicativos y afectivos de los hombres. 
 
 
 \item Vínculos extra-comunitarios organizacionales 
 
 Uno de los retos es que hay una variación sustancial entre los campos respecto de los tipos de vínculos externos que van a ser importantes predictores de la riqueza de los hogares. 
 
 Dejamos que los etnógrafos determinen qué preguntas deben hacerse. Pero solicitamos que todos los etnógrafos pregunten por los vínculos de los miembros de la familia con organizaciones externas, ya sean gubernamentales o no gubernamentales. 
 
 \item Otros vínculos extra-comunitarios
\end{enumerate}


\end{document}
